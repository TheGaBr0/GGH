\subsection{Struttura del progetto}
\label{sec:struttura}
Essendo Python il linguaggio scelto è conseguente che la struttura del progetto più corretta
si configuri come un pacchetto Python. Un pacchetto Python è una struttura organizzativa che 
racchiude moduli e sottopacchetti correlati, presentandosi come una directory nel filesystem.
Questa directory contiene file Python (.py) che fungono da moduli, un file speciale chiamato 
\_\_init\_\_.py che identifica la directory come pacchetto, e può includere altre subdirectory 
rappresentanti sottopacchetti. 
Il file \_\_init\_\_.py, pur potendo essere vuoto, è fondamentale per segnalare a Python che la 
directory deve essere trattata come un pacchetto, consentendo così un'importazione e un 
utilizzo strutturato dei componenti software all'interno del progetto.\\

\begin{figure}[h]
    \centering
    \begin{minipage}{7cm}
    \dirtree{%
    .1 GGH\_crypto.
    .2 \_\_init\_\_.py.
    .2 GGH.
    .3 \_\_init\_\_.py.
    .3 GGH.py.
    .2 GGH\_HNF.
    .3 \_\_init\_\_.py.
    .3 GGH\_HNF.py.
    .2 Utils.
    .3 \_\_init\_\_.py.
    .3 Utils.py.
    }
    \end{minipage}
    \caption{Struttura logica del pacchetto Python}
    \label{fig:pythonpackage}
\end{figure}

Come osservabile in Figura \ref{fig:pythonpackage}, il progetto è strutturato da:
\begin{itemize}
    \item \texttt{GGH\_crypto}: Rappresentante il pacchetto principale contenente tutte le implementazioni. 
    Esso è il modulo primario dal quale tutte le funzionalità dei
    sottopacchetti al suo interno possono essere chiamate e usate. 
    \item \texttt{GGH}: Sottopacchetto contenente l'implementazione di GGH.
    \item \texttt{GGH\_HNF}: Sottopacchetto contenente l'implementazione di GGH-HNF.
    \item \texttt{Utils}: Sottopacchetto contenente degli algoritmi relativi ai reticoli e 
    dei metodi in comune utilizzati dalle implementazioni dei due crittosistemi. 
\end{itemize}

La scelta di organizzare il progetto in tre sottopacchetti distinti, ciascuno con il proprio 
file di inizializzazione, invece di utilizzare tre moduli nel pacchetto principale \texttt{GGH\_crypto}, 
potrebbe essere considerata non ottimale. 
Tuttavia, questa organizzazione presenta vantaggi significativi: consente una chiara 
separazione logica dei componenti, offre maggiore flessibilità e semplifica la 
gestione delle importazioni tra i vari elementi. Di contro, è innegabile che 
questa configurazione comporti una maggiore complessità nella lettura del progetto 
a causa del numero più elevato di files.