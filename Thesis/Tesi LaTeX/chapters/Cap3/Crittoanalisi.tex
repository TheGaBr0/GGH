\section{Crittoanalisi di GGH}
In questa sezione saranno esaminate le vulnerabilità di GGH e gli attacchi derivanti da esse. 
I principali attacchi a cui GGH è soggetto includono:
\begin{itemize}
\item Computazione di una chiave privata: eseguendo una riduzione della base pubblica $\mathbf{B}$
si tenta di ottenere una chiave privata $\mathbf{B}'$ di qualità pari o simile a quella originale. 
\item Risoluzione diretta del CVP: tentare di risolvere il CVP del testo cifrato $\mathbf{c}$ 
rispetto al reticolo definito dalla base pubblica $\mathbf{B}$. 
\item Attacco di Nguyen: sfruttando la particolare struttura del vettore di errore $\mathbf{e}$ 
adottata dagli autori del crittosistema è possibile ricondursi ad un'instanza del CVP molto più 
semplice di quella proposta da GGH. 
\end{itemize}

\subsection{Crittoanalisi originale}
L'attacco più ovvio e semplice tra quelli proposti è la computazione di una chiave privata per invertire la
funzione trapdoor. Uno studio dettagliato e combinato con esperimenti pratici ha portato però 
gli autori a considerarlo inefficace per una dimensione maggiore di 100.
Un miglioramento dell'attacco precedente consiste nell'utilizzo di uno degli algoritmi
per approssimare il CVP presentati nella Sezione \ref{sec:babai}, tuttavia tale miglioramento
non permette a questo attacco di funzionare correttamente oltre la dimensione 150.
Gli autori, basandosi su quanto descritto finora, hanno ipotizzato che se l'algoritmo di 
riduzione utilizzato è LLL, il loro schema risulti sicuro per dimensioni superiori a 150. 
Tuttavia, poiché esistono algoritmi di riduzione migliori (Sezione \ref{sec:LLL-variants}), 
la loro conclusione è che la funzione trapdoor di GGH dovrebbe essere sicura per dimensioni 
comprese tra 250 e 300. \\
Di seguito viene presentato un esempio in dimensione 3 dell'attacco basato su 
risoluzione diretta del CVP. 
Per eseguire tale attacco sono stati utilizzati l'algoritmo LLL e la tecnica di incorporamento. 
Nonostante BKZ sia l'opzione più efficace, la bassa 
dimensionalità del problema rende i risultati ottenuti con LLL molto simili se non uguali. 
Pertanto, per semplicità, è stato scelto l'algoritmo LLL.


\begin{exmp} (Esempio di attacco a GGH con risoluzione diretta del CVP) \\
Siano $(\mathbf{B}, \sigma)$ e $\mathbf{c}$ rispettivamente chiave pubbblica e testo cifrato
utilizzati tra Alice e Bob nell'esempio \ref{exp:GGH}. Supponiamo che Eve abbia intercettato 
il testo cifrato e la chiave pubblica, e stia cercando di attaccare il criptosistema GGH
risolvendo direttamente il CVP.\\ Decide di procedere tramite tecnica di incorporamento
costruendo quindi la seguente matrice:
\begin{equation*}
    \mathbf{M} =
    \begin{bmatrix*}[l]
        -41335 & -110 & 374 & -913\\
        -11937 & -28 & 105 & -257\\
        -27073 & -69 & 242 & -592 \\
        \phantom{-}1 & \phantom{-}0 & \phantom{-}0 & \phantom{-}0
    \end{bmatrix*}.
\end{equation*}
Come secondo passaggio riduce $\mathbf{M}$ tramite LLL:
\begin{equation*}
    \mathbf{M}^* =
    \begin{bmatrix*}[l]
        1 & \phantom{-}3  &           -2 & \phantom{-}5\\
        1 &          -3   & \phantom{-}3 & \phantom{-}3\\
        1 &          -3   &           -4 & \phantom{-}2 \\
        4 & \phantom{-}1  & \phantom{-}3 & -2
    \end{bmatrix*}.
\end{equation*}
Eve a questo punto, sapendo che $\sigma = 3$, cerca di trovare $\mathbf{e}$. 
Per fare ciò seleziona i primi $n$ valori del vettore colonna  di $\mathbf{M}^*$ 
che abbia forma

\begin{equation*}
    \begin{bmatrix*}[l]
        \pm\sigma_1 \\
        \phantom{\pm}\vdots \\
        \pm\sigma_n \\
        \phantom{\pm}1
    \end{bmatrix*}
    =
    \begin{bmatrix*}[l]
        \phantom{-}3\\[0.1cm]
                  -3\\[0.1cm]
        \phantom{-}3\\[0.1cm]
        \phantom{-}1\\
    \end{bmatrix*}
    \ \text{ con conseguente } \
    \mathbf{e} =
    \begin{bmatrix*}[l]
        \phantom{-}3\\[0.1cm]
                  -3\\[0.1cm]
        \phantom{-}3\\[0.1cm]
    \end{bmatrix*}.
\end{equation*}
Come penultimo passaggio Eve calcola il vettore $\mathbf{w}$ più vicino a $\mathbf{c}$
\begin{equation*}
    \mathbf{w} = \mathbf{c} - \mathbf{e} =
    \begin{bmatrix*}[l]
        -41335\\
        -11937\\
        -27073
    \end{bmatrix*} -
    \begin{bmatrix*}[l]
        \phantom{-}3\\
                  -3\\
        \phantom{-}3\\
    \end{bmatrix*} =
    \begin{bmatrix*}[l]
        -41338\\
        -11934\\
        -27070
    \end{bmatrix*} 
\end{equation*}
e ottiene infine il messaggio originale $\mathbf{m}$ tramite
\begin{equation*}
    \mathbf{m} = \mathbf{B}^{-1} \mathbf{w} =
    \begin{bmatrix*}[l]
        -90\\
        \phantom{-}112\\
        \phantom{-}102
    \end{bmatrix*}.
\end{equation*}
\end{exmp}
\subsection{Attacco di Nguyen}