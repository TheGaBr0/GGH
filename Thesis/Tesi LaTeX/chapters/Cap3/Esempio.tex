\subsection{Esempio pratico}
\label{exp:GGH}
Prima di affrontare le varie tipologie di attacchi a GGH, viene mostrato un semplice esempio
(a dimensione 3) di come due entità, rispettivamente Alice e Bob, possano utilizzare questo
crittosistema per scambiare messaggi. 
\begin{exmp} (Esempio di funzionamento di GGH)
Sia $\mathbf{R}$ la base privata di Alice definita come:
\begin{equation*}
    \mathbf{R} =
    \begin{bmatrix*}[l]
        \phantom{-}11 & 0 & 0\\
                   -3 & 9 & 1\\
                   -4 & 1 & 8
    \end{bmatrix*}
    \ \text{ con } \mathcal{H}(\mathbf{R}) = 0.96022
\end{equation*}
Alice procede col generare la sua base pubblica $\mathbf{B}$ moltiplicando $\mathbf{R}$ con una matrice
unimodulare casuale $\mathbf{U}$:
\begin{equation*}
    \mathbf{U} =
    \begin{bmatrix*}[l]
        -10 & 34 & -83\\
        -5  & 18 & -44\\
        -13 & 45 & -110
    \end{bmatrix*}
    \ \text{ quindi } \mathbf{B}=\mathbf{R}\mathbf{U} =
    \begin{bmatrix*}[l]
        -110 & 374 & -913\\
        -28  & 105 & -257\\
        -69  & 242 & -592
    \end{bmatrix*}.
\end{equation*}
E' possibile osservare come $\mathbf{B}$ abbia un rapporto di Hadamard molto basso, più 
precisamente $\mathcal{H}(\mathbf{B}) = 0.02257$. Infine, utilizzando $\sigma = 3$, 
Alice compone le sue due chiavi
\begin{gather*}
    \mathbf{K}_{private} = (\mathbf{R}^{-1}, \mathbf{T}) 
    \ \text{ e } \  
    \mathbf{K}_{public} = (\mathbf{B}, \sigma)\\
    \text{con } \ \mathbf{T} = \mathbf{B}^{-1}\mathbf{R}.
\end{gather*}
Bob decide di mandare un messaggio $\mathbf{m} = [-90, 112, 102]^T$ con vettore di errore
$\mathbf{e} = [3, -3, -3]^T$. Utilizza quindi la chiave pubblica di Alice e ottiene il 
corrispondente testo cifrato
\begin{equation*}
    \mathbf{c} =
    \begin{bmatrix*}[l]
        -110 & 374 & -913\\
        -28  & 105 & -257\\
        -69  & 242 & -592
    \end{bmatrix*}
    \begin{bmatrix*}[l]
              -90 \\
    \phantom{-}112\\
    \phantom{-}102
    \end{bmatrix*}
    +
    \begin{bmatrix*}[l]
    \phantom{-}3\\
              -3\\
              -3
    \end{bmatrix*}
    =
    \begin{bmatrix*}[l]
        -41335\\
        -11937\\
        -27073
    \end{bmatrix*}.
\end{equation*}
Alice, una volta ricevuto il messaggio cifrato, è in grado di decifrarlo in maniera efficiente
usando la sua chiave privata. Infatti, avendo a disposizione
\[
    \mathbf{R}^{-1} = 
    \begin{bmatrix*}[l]
        \frac{1}{11}    & 0             & 0\\[6pt]
        \frac{20}{781}  & \frac{8}{71}  & \frac{-1}{71}\\[6pt]
        \frac{3}{71}    & \frac{-1}{71} & \frac{9}{71}
    \end{bmatrix*}
    \ \text{ e } \ 
    \mathbf{T} =
    \begin{bmatrix*}[l]
        0  & 5  & -2 \\
        22 & 21 & -25 \\
        9  & 8  & -10
    \end{bmatrix*}
\]
Alice, ottiene il messaggio originale calcolando 
\[
    \mathbf{x} = \lfloor \mathbf{R}^{-1}\mathbf{c}\rceil = 
    \begin{bmatrix*}[l]
        -3758 \\
        -2022 \\
        -5010 
    \end{bmatrix*}
    \ \text{ e } \ \mathbf{m} = \mathbf{T}\mathbf{x} = 
    \begin{bmatrix*}[l]
        -90 \\
        \phantom{-}112 \\
        \phantom{-}102
    \end{bmatrix*}.
\]
Si supponga ora che ci sia una terza persona, chiamata Eve, in ascolto nel canale di 
comunicazione tra Alice e Bob. Eve riesce ad ottenere la chiave pubblica di Alice e il 
messaggio cifrato inviato da Bob. Decide quindi di provare a decifrarlo usando la base 
pubblica invece della privata. 
Dato che non è in possesso della chiave privata di Alice, Eve tenterà la 
decifrazione usando l'algoritmo originale di Babai, riscrivendo poi il risultato come 
combinazione lineare della base pubblica. 
\\
Come primo passo Eve trova il vettore $\mathbf{w}$
più vicino a $\mathbf{c}$ usando come base $\mathbf{B}$:
\[
    \mathbf{w} = \mathbf{B} \lfloor \mathbf{B}^{-1}\mathbf{c}\rceil = 
    \begin{bmatrix*}[l]
        -41349 \\
        -11942 \\
        -27083 
    \end{bmatrix*}
\]
infine ottiene il messaggio $\mathbf{m}'$ computando
\[
    \mathbf{m}' = \mathbf{B}^{-1}\mathbf{w} = 
    \begin{bmatrix*}[l]
        -91 \\
        -119 \\
        -105 
    \end{bmatrix*}
\]
che, come si può notare, non è uguale al messaggio originale $\mathbf{m}$ inviato da Bob. 
\end{exmp}