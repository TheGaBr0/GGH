\section{Struttura e funzionamento di GGH}
\label{sec:gghintroduction}
Nel 1996, Oded Goldreich, Shafi Goldwasser e Shai Halevi\cite{GGH97} hanno introdotto un nuovo 
sistema crittografico a chiave pubblica basato sulla difficoltà di risolvere CVP in reticoli
di dimensioni elevate. 
\\
L'idea dietro GGH è la seguente: si supponga di avere un messaggio $\mathbf{m}$ 
codificato in un vettore appartenente ad un reticolo $\mathcal{L}$, un vettore target 
$\mathbf{t} \notin \mathcal{L}$ vicino ad $\mathbf{m}$ e
due basi $\mathbf{R}$ e $\mathbf{B}$ entrambe generanti $\mathcal{L}$
e rappresentanti rispettivamente base privata e base pubblica. 
Siano $\mathbf{R}$ una base buona e $\mathbf{B}$ una base cattiva, allora, tramite l'utilizzo
dell'algoritmo di arrotondamento di Babai (Sezione \ref*{sec:babai}), sarà possibile ritrovare
il vettore più vicino a $\mathbf{t}$ (che risulterà essere $\mathbf{m}$) usando la base privata,
ma non usando la base pubblica.

Più formalmente, GGH è definito da una funzione trapdoor (ovvero una funzione matematica che è facile da 
calcolare in una direzione, ma molto difficile da invertire senza dei dati segreti), la quale
è composta da 4 funzioni probabilistiche di complessità polinomiale:
\begin{itemize}
    \item Generate: Dato in input un intero positivo $n$ vengono generate due basi 
    $\mathbf{R} \text{ e } \mathbf{B}$ di rango massimo in
    $\mathbb{Z}^n$ e un numero positivo reale $\sigma$. Le basi $\mathbf{R} \text{ e } \mathbf{B}$
    sono rappresentate da matrici $n \times n$ e sono rispettivamente denominate base 
    privata e base pubblica. Sia $\mathbf{R}$ che $\mathbf{B}$ generano lo stesso reticolo 
    $\mathcal{L}$ e, insieme a $\sigma$, danno origine a chiave privata e chiave pubblica.
    Per maggiori dettagli riguardo la generazione delle chiavi si veda la prossima sezione.
    \item Sample: Dati in input $\mathbf{B}, \sigma$ vengono originati i vettori
    $\mathbf{m},\mathbf{e} \in \mathbb{R}^n$. \\
    Il vettore $\mathbf{m}$ viene scelto casualmente da un cubo in $\mathbb{Z}^n$ che sia 
    sufficientemente grande. Gli autori suggeriscono di scegliere in maniera casuale
    ogni valore di $\mathbf{m}$ uniformemente dall'intervallo $[-n^2, -n^2 + 1, \dots, +n^2]$, 
    sottolineando però che la scelta di $n^2$ è arbitraria e che non hanno prove
    di come essa possa influenzare la sicurezza del crittosistema stesso. Un intervallo
    sufficientemente grande che viene normalmente utilizzato è $[-128, 127]$. \\
    Il vettore $\mathbf{e}$ invece, viene scelto casualmente in  $\mathbb{R}^n$ in modo
    tale la media dei valori sia zero e la varianza sia $\sigma^2$. Il metodo più semplice
    per generare tale vettore è quello di scegliere ogni valore di $\mathbf{e}$ come $\pm\sigma$
    con probabilità $\frac{1}{2}$. Questo vettore ha l'importante funzione di essere un errore
    che viene aggiunto al calcolo del testo cifrato per complicarne la decifratura. 
    \item Evaluate: Dati in input $\mathbf{B}, \sigma, \mathbf{m}, \mathbf{e}$ si calcola
    $\mathbf{c} = \mathbf{m}\mathbf{B} + \mathbf{e}$. Con questo calcolo si ottiene
    il messaggio cifrato che è rappresentato da $\mathbf{c}$. 
    \item Invert: Dati in input $\mathbf{R},\mathbf{c}$ si utilizza la tecnica di arrotondamento
    di Babai per invertire la funzione trapdoor e ricavare il messaggio originale.
\end{itemize}

%
%			CHIAVI GGH
%
\subfile{Generate.tex}
%
%			ESEMPIO GGH
%
\subfile{Esempio.tex}