\section{Problemi sui reticoli}
\label{sec:problemi}
L’utilizzo della crittografia basata su reticoli parte dall'assunto che, soprattutto nei 
casi di spazi multidimensionali, la complessità computazionale derivante da determinati 
problemi su di essi, sia un limite invalicabile. I problemi reticolari 
più conosciuti e usati in ambito crittografico sono i seguenti:
\begin{itemize}
    \item Problema del Vettore più Corto (SVP): Data una base $\mathbf{B}$ di un reticolo, 
    trovare il vettore non nullo di lunghezza minima in $\mathcal{L}(\mathbf{B})$.
    \item Problema del Vettore più Vicino (CVP): Data una base $\mathbf{B}$ di un reticolo e un 
    vettore target $\mathbf{t}$ (non necessariamente nel reticolo), trovare il vettore 
    $\mathbf{w} \in \mathcal{L}(\mathbf{B})$ più vicino a $\mathbf{t}$ minimizzando 
    $\|\mathbf{t}-\mathbf{w}\|_2$.
    \item Problema dei Vettori Lineramente Indipendenti più Corti (SIVP): 
    Data una base $\mathbf{B} \in \mathbb{Z}^{n\times n}$ di un reticolo , 
    trovare $n$ vettori linearmente indipendenti 
    $\ {\mathbf{S} = [\mathbf{s}_1, \dots, \mathbf{s}_n]}$ 
    \mbox{(dove  $\mathbf{s}_i \in \mathcal{L}(\mathbf{B})$)}
    per tutte le $i$) minimizzando la quantità 
    ${||\mathbf{S}|| = \text{max}_i||\mathbf{s}_i||_2}$. 
    SIVP è una variante di SVP, ma a differenza di quest'ultimo, SIVP mira a identificare 
    un insieme di vettori indipendenti che siano i più corti possibile, in altre parole 
    la ricerca di una base ortogonale o ortonormale che generi il reticolo e che minimizzi
     la lunghezza dei suoi vettori.
    \end{itemize}

La complessità per risolvere CVP è stata provata essere NP-difficile\cite{CVP-NP09}, stessa
cosa vale per SVP, ma sotto alcune circostanze specifiche\cite{SVP-NP02}. Per questi motivi vengono
comparati come problemi dalla stessa difficoltà anche se, in pratica, risolvere CVP è considerato 
essere un po' più difficile di SVP a parità di dimensione.
Ognuno di questi due problemi ha un relativo sotto-problema che nient'altro è che una 
variante approssimativa: il Problema del Vettore più Vicino Approssimato (apprCVP) 
e Problema del Vettore più Corto Approssimato  (apprSVP). 
Questi sotto-problemi sono riferibili alla necessità di trovare un vettore non nullo la 
cui lunghezza sia maggiore di un fattore dato $\Psi(n)$, rispetto ad un vettore non nullo 
corretto che risulti essere più corto o più vicino, a seconda del problema.
Nel contesto di questa tesi, l'attenzione sarà focalizzata principalmente su questi 
sotto-problemi. Nello specifico apprCVP assume un ruolo di primaria importanza, essendo cruciale sia per 
il funzionamento del crittosistema GGH, oggetto centrale di questo studio, sia per la sua 
crittoanalisi. D'altra parte, apprSVP trova la sua applicazione esclusivamente nell'ambito
della crittoanalisi e sarà quindi trattato con minor dettaglio. 
Per affrontare quest'ultimo, si ricorre tipicamente a algoritmi di riduzione dei 
reticoli, un tema che sarà esaminato in dettaglio nella prossima sezione.
È importante precisare che, quando si parla di risolvere CVP o SVP, si intende sempre anche la 
risoluzione dei loro sottoproblemi. Nel resto di questa tesi quindi, la risoluzione 
del CVP e dell'SVP, sarà una generalizzazione che implicherà sempre anche la risoluzione delle loro versioni 
approssimate.