\section{Problemi sui reticoli}
\label{problemi}
Le costruzioni crittografiche basate su reticoli si basano sulla complessità 
computazionale che certe operazioni su di essi possono richiedere, 
soprattutto nel caso di spazi multidimensionali. I problemi computazionali 
più conosciuti relativi ai reticoli sono i seguenti:
\begin{itemize}
    \item Problema del Vettore più Corto (SVP): Data una base di un reticolo $\mathbf{B}$, 
    trovare il vettore non nullo di lunghezza minima in $\mathcal{L}(\mathbf{B})$.
    \item Problema del Vettore più Vicino (CVP): Data una base di un reticolo $\mathbf{B}$ e un 
    vettore target $\mathbf{t}$ (non necessariamente nel reticolo), trovare il vettore 
    $\mathbf{w} \in \mathcal{L}(\mathbf{B})$ più vicino a $\mathbf{t}$ minimizzando 
    $\|\mathbf{t}-\mathbf{w}\|_2$.
    \item Problema dei Vettori lineramente Vndipendenti più Vorti (SIVP): Data una base di un reticolo 
    $\mathbf{B} \in \mathbb{Z}^{n\times n}$, trovare $n$ vettori linearmente indipendenti 
    $\ {\mathbf{S} = [\mathbf{s}_1, \dots, \mathbf{s}_n]}$ 
    \mbox{(dove  $\mathbf{s}_i \in \mathcal{L}(\mathbf{B})$)}
    per tutte le $i$) minimizzando la quantità 
    ${||\mathbf{S}|| = \text{max}_i||\mathbf{s}_i||}$. 
    SIVP è una variante di SVP, ma a differenza di quest'ultimo, SIVP mira a identificare 
    un insieme di vettori indipendenti che siano i più corti possibile, in altre parole 
    la ricerca di una base ortogonale o ortonormale che generi il reticolo e che minimizzi
     la lunghezza dei suoi vettori.
    \end{itemize}

La complessità per risolvere CVP è stata provata essere NP-difficile\cite{CVP-NP09}, stessa
cosa vale per SVP sotto alcune circostanze specifiche\cite{SVP-NP02}, per questo vengono
comparati come problemi dalla stessa difficoltà anche se in pratica risolvere CVP è considerato 
essere un po' più difficile di SVP nella stessa dimensione. 
Ognuno di questi due problemi ha un relativo sotto-problema che nient'altro è che una 
variante approssimativa: il Problema del Vettore più Vicino Approssimato (apprCVP) 
e Problema del Vettore più Corto Approssimato  (apprSVP). Questi sotto-problemi consistono nel 
trovare un vettore non nullo la cui lunghezza non sia più lunga di un fattore dato $\Psi(n)$
rispetto a un vettore non nullo corretto (più corto o più vicino a seconda del problema).
\\
In particolare GGH si basa sulla risoluzione del CVP basandosi su una delle proprietà
fondamentali dei reticoli: la possibilità di usare più basi per lo stesso reticolo.
Utilizzando due basi $\mathbf{A}$ e $\mathbf{B}$, definite rispettivamente come 
"buona" e "cattiva", ma che generano lo stesso reticolo, diventa più agevole 
risolvere determinati problemi sui reticoli utilizzando la base $\mathbf{A}$ piuttosto 
che con $\mathbf{B}$. 
Per questi motivi il CVP sarà il fulcro dei problemi discussi in questa tesi assieme al SVP,
il quale verrà trattato prevalentemente per quanto riguarda la crittoanalisi di GGH. 