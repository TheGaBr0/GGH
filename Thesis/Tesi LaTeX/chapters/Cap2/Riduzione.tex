\section{Riduzione di un reticolo}
\subsection{Rapporto di Hadamard}
\label{sec:hadamard}
Si supponga di avere a disposizione due basi $\mathbf{A}$ e $\mathbf{B}$ che godono della
proprietà di generare lo stesso reticolo. Seppur condividendo tale caratteristica, 
$\mathbf{A}$ e $\mathbf{B}$ sono in realtà molto diverse nella loro struttura; in particolare
$\mathbf{A}$ è composta da vettori corti e quasi ortogonali fra loro mentre $\mathbf{B}$ è
composta da vettori lunghi e quasi paralleli fra loro.\\
La qualità di una base risiede in queste differenze dei vettori costituenti le basi, chiamiamo
quindi base "buona" $\mathbf{A}$ e base "cattiva" $\mathbf{B}$.
E' necessario però definire una metrica per valutare quanto una base sia buona o meno; a tal
proposito Hadamard\cite{HDMRD08} introdusse una formula quantitativa per misurare la qualità
di una base reticolare, il cosiddetto rapporto di Hadamard. \\

Data una base $\mathbf{B} = [\mathbf{b_1}, \mathbf{b_2}. \dots, \mathbf{b}_n]$ e un reticolo
$\mathcal{L}$ di dimensione $n$ generato da $\mathbf{B}$, il rapporto di Hadamard della base
$\mathbf{B}$ è definito dal valore:
\[
\mathcal{H}(\mathbf{B}) = 
    \Biggl( 
        \frac{\det(\mathcal{L})}
        {\|\mathbf{b}_1\|_2 \cdot \|\mathbf{b}_2\|_2, \cdots \|\mathbf{b}_n\|_2} 
    \Biggr)^{\frac{1}{n}}
\]
\\
Il rapporto di Hadamard si configura nell'intervallo $(0, 1]$, dove più vicino all'1 si è 
e più la base è buona, viceversa più vicino allo 0 si è e più la base è cattiva. Questa formula
verrà utilizzata come unica misura per verificare la qualità degli esempi di basi che verranno
presentate più avanti in questa tesi.

\subsection{Ortogonalizzazione Gram-Schmidt}
 
Ora che è possibile giudicare una base dato il suo rapporto di Hadamard, utilizziamo la base
$\mathbf{B}$ definita nella precedente sezione, la quale ipotizziamo abbia un 
$\mathcal{H}(\mathbf{B})$ prossimo allo zero. Se volessimo utilizzare questa base per risolvere
uno dei problemi dei reticoli, molto probabilmente non riusciremmo mai a raggiungere una soluzione
che sia valida o quantomeno che sia vicina alla soluzione ottima.
A questo proposito sono stati ideati degli algoritmi in grado di ortogonalizzare una base
cattiva per convertirla in una buona e mantenere le proprietà del reticolo iniziale, si
ottiene quindi una base $\mathbf{B'}$ tale che: 
$\mathcal{H}(\mathbf{B}^*) \approxeq 1$ e che $\det(\mathbf{B}) = \det(\mathbf{B}^*)$.
Nell'ambito della riduzione di reticoli è importante notare come il gap di un reticolo giochi
un ruolo chiave: è noto che più il gap è grande e più la riduzione è semplice.\\

Prima di discutere questo tipo di algoritmi è necessario affrontare brevemente 
l'algoritmo di Gram-Schmidt, il quale, esegue un tipo di ortogonalizzazione che viene applicata
su spazi vettoriali e che è anche chiamata Ortogonalizzazione Gram-Schmidt (GSO).
Questo algoritmo non è adottabile direttamente sulle basi reticolari in quanto esso andrebbe a 
violare la proprietà dei coefficienti integrali, di fondamentale importanza nella
definizione di reticolo. Nonostante ciò, questo algoritmo gode di una proprietà chiave che
viene utilizzata in algoritmi di riduzione dei reticoli. Come dimostrato in~\cite[Teorema 7.13]{HDMRD08}:

siano span$(\mathbf{b}_1, \mathbf{b}_2, \dots, \mathbf{b}_n)$ e 
span$(\mathbf{b}_1^*, \mathbf{b}_2^*, \dots, \mathbf{b}_n^*)$ 
gli spazi vettoriali generati rispettivamente dalle righe di $\mathbf{B}$ e $\mathbf{B}^*$,
allora se $\mathbf{B}^*$ è il risultato di GSO applicato a $\mathbf{B}$:
\[
\text{span}(\mathbf{b}_1, \mathbf{b}_2, \dots, \mathbf{b}_n) =
\text{span}(\mathbf{b}_1^*, \mathbf{b}_2^*, \dots, \mathbf{b}_n^*).
\]
Quindi facendo uso del GSO come subroutine di un algoritmo per la trasformazione delle 
matrici di spazi vettoriali, si ottiene una importante riduzione del costo computazionale 
e nel contempo, si semplifica l'implementazione di algoritmi per la riduzione di reticoli.

\begin{algorithm}
    \caption{Algoritmo di Gram-Schmidt}
    \label{alg:one}
    \large
    \DontPrintSemicolon
    \KwIn{Una matrice $\mathbf{B}$ tale che $\text{rango}(\mathbf{B}) = \text{righe}(\mathbf{B})$}
    \KwOut{Una matrice $\mathbf{B}^*$ ortogonale}
    $\mathbf{b}_1^* = \mathbf{b}_1$\;
    \For{$i = 2$ \KwTo$n$}{
        \For{$j = 1$ \KwTo$i-1$}{
            $\mu_{i,j} = \frac{\langle \mathbf{b}_i,\mathbf{b}_j^* \rangle}
            {\langle \mathbf{b}_j^*,\mathbf{b}_j^* \rangle}$\;
        }
        $\mathbf{b}_i^* = \mathbf{b}_i - \sum_{j=1}^{i-1}\mu_{i,j}\mathbf{b}_j^*$\;
    }
    \Return{$\mathbf{B}^*$}
\end{algorithm}
Dove rango() indica il rango e righe() il numero delle righe. 

\subsection{Algoritmo di Lenstra-Lenstra-Lovász}
L'algoritmo di Lenstra-Lenstra-Lovász (LLL)\cite{LLL82,HDMRD08} è noto come uno dei più famosi algoritmi per 
la riduzione dei reticoli. 
In teoria, opera con un tempo polinomiale $O(n^6 (\log\mathcal{E})^3)$, dove $n$ è la dimensione di un 
reticolo $\mathcal{L}$ dato ed $\mathcal{E}$ rappresenta la massima lunghezza euclidea dei vettori nella base 
fornita. Il risultato di LLL è una base ridotta, il cui vettore più breve può essere 
considerato una soluzione approssimata dell'apprSVP con un fattore di $2^{(n-1)/2}$ rispetto 
alla lunghezza del vettore corretto più corto. Nonostante questo fattore sia 
esponenziale, l'algoritmo si dimostra sorprendentemente efficiente nella pratica e ciò lo rende
un tassello fondamentale negli algoritmi di risoluzione del CVP e del SVP. 

Una base $\mathbf{B}^*$ per essere considerata LLL-ridotta deve soddisfare due condizioni:
\begin{itemize}
    \item Condizione di grandezza: $\mu_{i,j} = \frac{\langle \mathbf{b}_i,\mathbf{b}_j^* \rangle}
    {\langle \mathbf{b}_j^*,\mathbf{b}_j^* \rangle}  
    \leq \eta$ per ogni $1\leq j < i \leq n$.
    \item Condizione di Lovász: $\langle \mathbf{b}_i^*,\mathbf{b}_i^* \rangle \geq 
    (\delta- \mu^2_{i,i-1}) \langle \mathbf{b}_{i-1}^*,\mathbf{b}_{i-1}^* \rangle$
    per ogni $1 < i \leq n$.
\end{itemize}
I valori $\delta$ e $\eta$ sono standardizzati rispettivamente a $\frac{3}{4}$ e $\frac{1}{2}$.

\begin{algorithm}
    \caption{Algoritmo di LLL, con $\delta = \frac{3}{4}, \eta=\frac{1}{2}$}
    \label{alg:two}
    \DontPrintSemicolon
    \KwIn{Una matrice $\mathbf{B}$ che sia base di un reticolo}
    \KwOut{Una matrice $\mathbf{B}^*$ LLL-ridotta}
    $k=1$\;
    $\mathbf{b}_1^* = \mathbf{b}_1$\;

    \While{$k<=n$}{
        \For{$j = k-1$ \KwTo$0$}{ 
            \If(\tcp*[f]{Condizione di grandezza}){$|\mu_{k,j}| > \eta$}{ 
                $\mathbf{b}_k = \mathbf{b}_k - \lfloor \mu_{k,j} \rceil \mathbf{b}_j$\;
                GramSchmidt($\mathbf{B}$)\; % chktex 36
         }
        }
        \If(\tcp*[f]{Condizione di Lovász})
        {$\langle \mathbf{b}_k^*,\mathbf{b}_k^* \rangle \geq 
        (\delta - \mu^2_{k,k-1}) \langle \mathbf{b}_{k-1}^*,\mathbf{b}_{k-1}^* \rangle$}{
            $k=k+1$\;
        }
        \Else{
            Scambia $\mathbf{b}_{k-1}$ e $\mathbf{b}_k$\;
            GramSchmidt($\mathbf{B}$)\; % chktex 36
            $k=\text{max}(k-1, 1)$
        }
    }
\end{algorithm}

\begin{exmp} (Riduzione di una base con LLL)
    Sia $\mathcal{L}$ un reticolo generato dalla matrice:
    \begin{equation*}
        \mathbf{B} =
        \begin{bmatrix*}
            \phantom{-}87634 & \phantom{-}88432 & \phantom{-}94345\\
            \phantom{-}32323 & \phantom{-}27883 & \phantom{-}40323\\
            -21221           & -11234           & -32123
        \end{bmatrix*}
    \end{equation*}
    \[
        \text{con } \mathcal{H}(\mathbf{B}) = 0.14318 \text{ e } \det(\mathbf{B}) = -1079906335101.
    \]
    Si applichi ora una riduzione LLL alla matrice $\mathbf{B}$:
    \begin{equation*}
        \mathbf{B}^* =
        \begin{bmatrix*}
            -784 & \phantom{-}632  & \phantom{-}2701\\
            -14823 & \phantom{-}9207          & -7717\\
            -13454 & -14753 & -97
        \end{bmatrix*}
    \end{equation*}
    Ricalcolando $\mathcal{H}(\mathbf{B}^*)$ è possibile osservare che:
    \[
        \mathcal{H}(\mathbf{B}^*) = 0.99442 \text{ e }\det(\mathbf{B}^*) = 1079906335101.
    \]
    E' stato ottenuto un incremento notevole del rapporto di Hadamard grazie alla riduzione LLL senza 
    interferire con le proprietà della base $\mathbf{B}$. Infatti $|\det(\mathbf{B})| = |\det(\mathbf{B}^*)|$ e,
    usando la formula descritta nella Sezione \ref{sec:reticoli}, 
    è possibile trovare una matrice di interi
    \begin{equation*}
        \mathbf{U} =
        \begin{bmatrix*}
            -1 & 4 & 2\\
            -6 & 25 & 14\\
            -3 & 11 & 5
        \end{bmatrix*}
        \ \text{ con } \det(\mathbf{U}) = -1
    \end{equation*}
    tale per cui $\mathbf{U}\mathbf{B} = \mathbf{B}^*$.
\end{exmp}

\subsection{Varianti di LLL}
\label{sec:LLL-variants}
LLL è un eccellente algoritmo in grado di restituire in un tempo polinomiale una matrice 
quasi ortogonale partendo da una con vettori quasi paralleli, o che comunque è ritenibile 
di bassa qualità. Esistono però varianti che ne velocizzano i calcoli, 
così come altri algoritmi capaci di restituire una base di qualità ancora superiore 
rispetto a quella ottenuta con la riduzione LLL. Di seguito vengono presentate 
brevemente tre versioni dell'algoritmo. 
\\
La prima versione discussa è quella in virgola mobile (FPLLL)\cite{FPLLL05}, la quale 
utilizza aritmetica in virgola mobile a 
precisione arbitraria per accelerare i calcoli razionali dell'algoritmo originale. 
Questa versione ha come vantaggio un aumento delle performance: in comparazione con LLL
il tempo di computazione nel caso peggiore è $O(n^3 (\log\mathcal{E})^2)$. E' importante notare che
l'utilizzo di aritmetica a virgola mobile per velocizzare i calcoli è una tenica comune e utilizzabile
per tutti gli algoritmi di riduzione dei reticoli spiegati in questa sezione.
\\ 
La seconda versione è stata presentata da Schnorr-Euchner~\cite{DEEPLLL94} ed il suo nome
originale è "deep insertions" ovvero inserzioni profonde. In LLL (Algoritmo~\ref*{alg:two}),
è presente un passaggio in cui avviene uno scambio tra il vettore 
$\mathbf{b}_{k-1}$ e $\mathbf{b}_{k}$, il quale di solito permette qualche riduzione di grandezza
ulteriore del nuovo $\mathbf{b}_{k}$. Nella variante deep insertions, viene invece inserito
$\mathbf{b}_{k}$ tra $\mathbf{b}_{i-1}$ e $\mathbf{b}_{i}$ con $i$ che viene scelta in modo
da apportare una maggiore riduzione di grandezza. L'algoritmo risultante, nel caso peggiore,
potrebbe non terminare in un tempo polinomiale, ma in pratica, quando eseguito sulla maggioranza
dei reticoli, termina rapidamente e può fornire in output una base ridotta 
significativamente migliore di quella di LLL standard. 
\\
L'ultima variante discussa è basata sull'algoritmo di riduzione 
Korkin–Zolotarev (KZ)\cite[sezione 18.5]{Galbraith18}. Le caratteristiche di una base KZ-ridotta
sono generalmente migliori rispetto a quelle di LLL, ma richiedono una complessità maggiore
e un tempo di computazione non polinomiale; per le proprietà complete si veda il riferimento. 
Più nel dettaglio, il problema principale, è che non esiste un algoritmo in grado di computare
una base KZ in tempo polinomiale. L'algoritmo più veloce conosciuto richiede un tempo
di computazione esponenziale rispetto alla dimensione. 
Per compensare a tale problema KZ apporta un 
grande vantaggio in rispetto all'accuratezza della riduzione, infatti, il primo vettore
di una base KZ-ridotta è sempre una soluzione al SVP. Dato che la complessità di KZ
cresce con $n$, è logico pensare che a basse dimensioni sia comunque sufficientemente
veloce. Un'idea è quindi quella di computare
una riduzione di proiezioni a dimensioni più basse del reticolo originale.
L'algoritmo in questa configurazione prende il nome di Korkine-Zolotarev a blocco (BKZ), 
il quale, se combinato con LLL, diventa una variante di quest'ultimo chiamata LLL-BKZ. Questa
variante è in grado di bilanciare costo computazionale e qualità di riduzione ottenendo così
l'algoritmo più efficiente per SVP in grandi dimensioni, dimostrando anche una qualità di riduzione
significativamente migliore di quella di LLL standard. \\

Per reticoli di dimensioni ancora maggiori, dove anche BKZ potrebbe risultare computazionalmente oneroso, 
è stata sviluppata una versione ulteriormente ottimizzata chiamata BKZ "pruned" o potata \cite{BKZPRUNED}. Questa variante 
mantiene l'efficacia di BKZ nel bilanciare costo computazionale e qualità della riduzione, 
ma introduce una tecnica di potatura nell'enumerazione dei vettori. Tale tecnica è spesso usata in 
informatica al fine di ottimizzare algoritmi riducendo lo spazio di ricerca, permettendo così di ottenere 
soluzioni approssimate (e solitamente corrette) in tempi significativamente minori rispetto all'esplorazione completa.