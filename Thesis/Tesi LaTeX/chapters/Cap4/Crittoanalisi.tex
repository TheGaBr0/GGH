\section{Limiti pratici di GGH-HNF}
\label{sec:HNFLimits}
GGH-HNF riesce a risolvere i problemi di GGH con successo, riuscendo a diventarne una variante migliorata
a tutti gli effetti. Nelle conclusioni di \cite{HNF01}, Micciancio suggerisce che una dimensione di 
500 potrebbe offrire un livello di sicurezza adeguato, mantenendo al contempo una dimensione delle 
chiavi accettabile grazie all'impiego della forma normale di Hermite. Sfortunatamente però le 
supposizioni di Micciancio si rivelarono troppo ottimistiche. \\
Nel Gennaio del 2004 Christoph Ludwig stilò un report tecnico \cite{HNF04} nel quale crittoanlizzò
GGH-HNF e ne testò i suoi limiti pratici. 

\subsubsection{Generazione delle chiavi}
Una serie di esperimenti vennero condotti sulla generazione delle chiavi di GGH-HNF.
Il lavoro si è concentrato su diverse dimensioni dei reticoli, fino a 475, 
con un caso speciale in dimensione 800.
Per le chiavi private, il processo più impegnativo è stato la riduzione LLL delle basi scelte 
casualmente. Secondo gli esperimenti di Ludwig questo ha richiesto fino a 58 minuti nelle dimensioni più alte.
Molto più pesante invece il dato riguardante la generazione della chiave pubblica: il miglior algoritmo
a disposizione impiegò 4 ore. Per quanto riguarda il caso in dimensione 800 i tempi rilevati furono
di 4 ore e mezza per la chiave privata e 46 ore per quella pubblica.

\subsubsection{Cifratura e decifratura}
Come descritto precedentemente, la particolare struttura della forma normale di Hermite consente una
cifratura molto veloce. Ciò venne confermato dai test di Ludwig i quali impiegarono in media solo $0.29$
secondi in dimensione 800. Le cose cambiano drasticamente con la decifratura: a causa dell'ortogonalizzazione
Gram-Schmidt e della precisione richiesta, lo spazio occupato e il tempo richiesto per i calcoli
cresce a livelli non accettabili. Gli esperimenti richiedettero 40 minuti per ortogonalizzare
e rispettivamente 13 e 73 minuti per decifrare in dimensione 475 e 800. E' però importante precisare
che Ludwig utilizzò il metodo del piano più vicino di Babai che restituisce una soluzione più precisa, 
ma contemporanemente richiede più tempo per trovarla a causa della sua complessità computazionale maggiore. 

\subsubsection{Attacchi a GGH-HNF}
Gli attacchi a GGH-HNF coinvolsero reticoli di dimensioni fino a 280, impiegando vettori di errore 
la cui lunghezza variava dal 10\% al 100\% del $\rho$. Gli algoritmi di riduzione impiegati spaziarono da
LLL a diverse varianti di BKZ. L'algoritmo LLL dimostrò efficacia in dimensione 280 con vettori di errore corti,
ma si rivelò inefficace per dimensioni pari o superiori a 180 con vettori più lunghi. 
L'incremento della dimensione del reticolo rese necessario l'impiego di BKZ con blocchi fino a 60 per
mantenere l'efficacia degli attacchi. Utilizzando una tecnica di estrapolazione basata sui risultati sperimentali
ottenuti, 
Ludwig è riuscito a prevedere l'efficacia degli attacchi su dimensioni più elevate, 
che non erano state direttamente testate. Questo ha fornito una visione sulla sicurezza futura del 
sistema. Considerando scenari di complessità esponenziale e subesponenziale, Ludwig ha suggerito 
che per garantire la sicurezza del GGH-HNF sarebbero necessarie dimensioni del reticolo di almeno 800, 
un valore significativamente superiore rispetto alle stime iniziali di Micciancio, 
che si attenevano su una dimensione di 500.\\

E' ovvio che i valori riportati da Ludwig portino alla comune considerazione che le basse performance
di GGH-HNF su alte dimensioni lo rendano praticamente non utilizzabile, soprattutto dopo che 
le estrapolazioni fatte hanno indicato la necessità di dimensioni superiori a 800. E' 
però importante precisare che, dato l'avanzamento della tecnologia, tali dati siano ormai obsoleti.
Come mostrano i dati sperimentali presentati in sezione \ref{label}(sezione futura), grazie ai nuovi
algoritmi è possibile generare e decifrare chiavi con molte meno risorse spaziali e
temporali di quelle richieste negli anni della pubblicazione dell'algoritmo, ormai venti anni fa. 