\section{Limiti pratici di GGH-HNF}
\label{sec:HNFLimits}
GGH-HNF riesce a risolvere i problemi di GGH con successo, diventando una variante migliorata
a tutti gli effetti. Nelle conclusioni di \cite{HNF01}, Micciancio suggerisce che una dimensione di 
500 potrebbe offrire un livello di sicurezza adeguato, mantenendo al contempo una dimensione delle 
chiavi accettabile grazie all'impiego della forma normale di Hermite. Sfortunatamente però le 
supposizioni di Micciancio si sono rivelate troppo ottimistiche. \\
Nel Gennaio del 2004 Christoph Ludwig stilò un report tecnico \cite{HNF04} nel quale crittoanlizzò
GGH-HNF e ne testò i suoi limiti pratici. 

\subsubsection{Generazione delle chiavi}
Una serie di esperimenti vennero condotti sulla generazione delle chiavi di GGH-HNF.
Il lavoro si è concentrato su diverse dimensioni dei reticoli, fino a 475, 
con un caso speciale in dimensione 800.
Per le chiavi private, il processo più impegnativo è stato la riduzione LLL delle basi scelte 
casualmente. Secondo gli esperimenti di Ludwig questo ha richiesto fino a 58 minuti nelle dimensioni più alte.
Molto più pesante invece il dato riguardante la generazione della chiave pubblica: il miglior algoritmo
a disposizione impiegò 4 ore. Per quanto riguarda il caso in dimensione 800 i tempi rilevati furono
di 4 ore e mezza per la chiave privata e 46 ore per quella pubblica.

\subsubsection{Cifratura e decifratura}
Come descritto precedentemente, la particolare struttura della forma normale di Hermite consente una
cifratura molto veloce. Ciò venne confermato dai test di Ludwig i quali impiegarono in media solo $0.29$
secondi in dimensione 800. Le cose cambiano drasticamente con la decifratura: a causa dell'ortogonalizzazione
Gram-Schmidt e della precisione richiesta, lo spazio occupato e il tempo richiesto per i calcoli
cresce a livelli non accettabili. Gli esperimenti richiedettero 40 minuti per ortogonalizzare
e rispettivamente 13 e 73 minuti per decifrare in dimensione 475 e 800. È però importante precisare
che Ludwig utilizzò il metodo del piano più vicino di Babai che restituisce una soluzione più precisa, 
ma contemporanemente richiede più tempo per trovarla a causa della sua complessità computazionale maggiore. 

\subsubsection{Attacchi a GGH-HNF}
Gli attacchi a GGH-HNF furono condotti su reticoli di dimensioni fino a 280, con vettori 
di errore di lunghezza variabile tra il 10\% e il 100\% del $\rho$. Gli attacchi furono
configurati con la tecnica di incorporamento, usando LLL, BKZ-20 e BKZ-60 potato. LLL 
si dimostrò efficace in 
dimensione 280 con vettori di errore corti, ma inefficiente per dimensioni da 180 in su 
con vettori più lunghi. L'aumento di lunghezza dei vettori in dimensioni più alte richiese 
necessario l'impiego 
di algoritmi più avanzati. Inizialmente, si tentò con BKZ-20, ma quando questo si rivelò 
insufficiente, si passò al più potente BKZ-60 potato.
Ludwig, estrapolando dai dati sperimentali, stimò l'efficacia degli attacchi su dimensioni 
maggiori che non erano state oggetto di verifica, offrendo una prospettiva sulla 
sicurezza futura del sistema. Considerando scenari di complessità esponenziale e 
subesponenziale, suggerì che per garantire la sicurezza del GGH-HNF sarebbero necessarie 
dimensioni del reticolo di almeno 800, ben oltre le stime iniziali di Micciancio di 500.
Questi risultati portarono alla conclusione che le scarse prestazioni di GGH-HNF su alte 
dimensioni lo rendessero impraticabile, specialmente considerando la necessità di 
dimensioni superiori a 800. Tuttavia, è cruciale notare che tali dati sono ormai 
superati dal progresso tecnologico. Come dimostrano i risultati sperimentali in Sezione 
\ref{sec:risultati_sicurezza}, gli algoritmi moderni consentono di generare e decifrare 
chiavi con risorse notevolmente inferiori rispetto a quelle richieste all'epoca della 
pubblicazione dell'algoritmo, circa vent'anni fa.