\section{Struttura e funzionamento di GGH-HNF}
\label{sec:ggh-hnf}
Sulla base di quanto precedentemente esposto nella Sezione \ref{sec:gghintroduction}, in GGH
il messaggio originale $\mathbf{m}$ viene codificato in un vettore $\mathbf{x}$ 
appartenente al reticolo, e il testo cifrato risulta come 
$\mathbf{c} = \mathbf{x}\mathbf{B} + \mathbf{e}$.
Le ottimizzazioni sviluppate da Micciancio hanno portato a una modifica di questo approccio. 
Invece di generare in maniera casuale sia il vettore $\mathbf{x}$ che la base $\mathbf{B}$, 
Micciancio ha scelto di codificare il messaggio direttamente nel vettore di errore $\mathbf{e}$, 
procedendo con un approccio deterministico per la generazione dei precedentemente citati parametri.
Questa scelta nasce dalla difficoltà nel generare vettori e basi random che abbiano una 
sicurezza intrinseca e dimostrabile. Questa difficoltà si ripercuote sulla sicurezza del 
crittosistema: $\mathbf{B}$ scelta casualmente rilascia spesso informazioni parziali
relative a $\mathbf{R}$ permettendo così una facile riduzione di essa. \\
Per superare questo problema, Micciancio decide di non generare più $\mathbf{B}$ tramite 
la costruzione di matrici unimodulari casuali moltiplicate per $\mathbf{R}$. 
Invece, opta per un approccio deterministico basato sulla forma normale di Hermite (HNF)
di $\mathbf{R}$. La forma normale di Hermite è una rappresentazione canonica e unica per 
una data matrice, ottenuta mediante operazioni elementari di riga e colonna. 
Essa presenta una struttura triangolare e garantisce che gli elementi sulla diagonale principale 
siano ordinati in modo decrescente.
Poiché l'HNF è unica per ogni reticolo la chiave pubblica $\mathbf{B}$ non rivela informazioni 
sulla chiave privata $\mathbf{R}$, se non il reticolo $\mathcal{L}$ che genera.
Inoltre, qualsiasi informazione su $\mathbf{R}$ che possa essere efficacemente calcolata 
da $\mathbf{B}$ può 
essere altrettanto efficacemente calcolata a partire da qualsiasi altra base $\mathbf{B}'$ 
che genera lo stesso reticolo $\mathcal{L}$. Questo perché 
$\mathbf{B}$ = HNF($\mathbf{R}$) = HNF($\mathbf{B}'$). \\
Ottenuto $\mathbf{B}$ è necessario quindi calcolare un vettore
$\mathbf{x}\mathbf{B}$ appartenente al reticolo che verrà poi aggiunto a $\mathbf{e}$ come
da equazione \ref{eq:GGHencryption}.  
L'idea migliore sarebbe scegliere il vettore in modo casuale e uniforme, ma questa scelta 
non è praticabile. Tuttavia, Micciancio in \cite[sezione 4.1]{HNF01} dimostra che tale 
risultato può essere ottenuto mediante il semplice calcolo di 
$\mathbf{x} = \mathbf{e} \bmod \mathbf{B}$. Quindi, invece di aggiungere a $\mathbf{e}$ un
vettore casuale $\mathbf{x}\mathbf{B}$, si riduce $\mathbf{e}$ modulo la base pubblica.
Data la particolare struttura della matrice 
$\mathbf{B}$ nella sua forma HNF, questo calcolo risulta particolarmente 
semplice da effettuare. Partendo da un vettore $\mathbf{x}$
inizialmente nullo, si può calcolare un valore di $\mathbf{x}$ alla volta, iniziando dall'ultimo
componente $\mathbf{x}_n$, tramite la seguente formula:
\begin{equation}
    \label{eq:HNFreduction}
    \mathbf{x}_i = \left\lfloor \frac{\mathbf{e}_i - 
    \sum_{j=i+1}^{n-1} \mathbf{B}_{j,i} \mathbf{x}_j}{\mathbf{B}_{i,i}} \right\rfloor
\end{equation}
e ottenere infine:
\begin{equation}
    \label{eq:HNFencryption}
    \mathbf{c} = \mathbf{e} - \mathbf{x}\mathbf{B}.
\end{equation}

Come si può notare le due equazioni \ref{eq:GGHencryption} e \ref{eq:HNFencryption} sono
diverse, ma come dimostrato in \cite[sezione 4.3]{HNF01} esse garantiscono lo stesso
livello di sicurezza. \\
Un ulteriore cambiamento, conseguente dalla scelta di Micciancio di usare $\mathbf{e}$ come vettore
rappresentante il messaggio, è la totale mancanza di un fattore di bilanciamento, ruolo che nella
versione originale del crittosistema veniva ricoperto da $\sigma$. 
Il processo di decifratura infatti, 
basato sulla tecnica di arrotondamento di Babai, rimane invariato. Pertanto, quanto detto in 
Sezione \ref{sec:gghkeys}, è ancora vero anche per GGH-HNF: il crittosistema è probabilistico 
e necessità di un parametro per bilanciarne la probabilità di decifratura con chiave pubblica e 
privata. 
Nella versione originale di GGH, $\sigma$, veniva derivato direttamente dalla base privata
e veniva utilizzato come parametro assoluto per la costruzione di $\mathbf{e}$. \\
In GGH-HNF invece, Micciancio, decide di creare un nuovo parametro $\rho$ derivandolo sempre
dalla base privata, ma con un approccio differente. 
La base privata $\mathbf{R}$ viene ortogonalizzata utilizzando l'algoritmo di Gram-Schmidt, 
producendo la base ortogonale $\mathbf{R}^*$. Successivamente $\rho$ è calcolato attraverso:
\begin{equation}
    \label{eq:rho}
    \rho = \frac{1}{2}min_i\|\mathbf{r}^*_i\|_2.
\end{equation}
$\rho$ rappresenta un raggio di correzione: se la lunghezza del vettore di errore 
è minore di questo raggio la decifratura avrà successo. 
Poiché la base privata è conosciuta esclusivamente dal destinatario, è essenziale che il parametro 
$\rho$ sia incluso nella chiave pubblica, insieme alla base pubblica $\mathbf{B}$. Questa inclusione è 
fondamentale affinché il mittente possa generare messaggi appropriati, codificandoli nel vettore di 
errore $\mathbf{e}$. In questo modo, il mittente può assicurarsi che i messaggi cifrati siano compatibili 
con i parametri di decifratura del destinatario, garantendo che possano essere decifrati correttamente 
utilizzando la base privata del ricevente.\\
Un'ultima modifica proposta riguarda la generazione di $\mathbf{R}$. Mentre GGH optava per la creazione 
di una matrice rettangolare successivamente moltiplicata per una matrice casuale, Micciancio suggerisce 
un metodo diverso basato sui suoi esperimenti.
Il nuovo approccio consiste nel generare direttamente una matrice casuale i cui elementi sono interi 
compresi nell'intervallo $[-n, \dots, n]$. A questa matrice viene poi applicata una riduzione LLL. 
Gli esperimenti provarono che questo metodo produce basi con un $\rho$ sufficientemente grande, più
precisamente $\rho = \frac{n}{2}$. 
%
%			ESEMPIO HNF
%
\subfile{Esempio.tex}