\chapter{Migliorare GGH usando la Forma Normale di Hermite}
\chaptermark{CRITTOSISTEMA A CHIAVE PUBBLICA GGH-HNF}
\label{cap:GGH-HNF}
%
%			CRITTOSISTEMA GGH-HNF
%
Considerando i vari attacchi discussi nella sezione \ref{sec:cryptoanalysis} e le note vulnerabilità 
del crittosistema GGH, è evidente che, per un suo utilizzo sicuro, la dimensione delle chiavi deve essere almeno superiore a 400. Tuttavia, una tale 
dimensione comporta complessità spaziali e temporali tali da rendere il crittosistema
poco competitivo rispetto ad altri attualmente in uso come RSA o DSS. Nel 2001, Daniele 
Micciancio \cite{HNF01} ha proposto una versione migliorata del crittosistema basata sulla 
forma normale di Hermite, nota come GGH-HNF. Questo schema mira ad aumentare sia le 
performance che la sicurezza in comparazione alle 
risorse necessarie rispetto alla versione originale di GGH.

\subfile{Introduzione.tex}
%
%           CRITTOANALISI GGH-HNF
%
\subfile{Crittoanalisi.tex}