\chapter{Introduzione}
\label{cap1}

Nell'era digitale in cui viviamo, la sicurezza delle informazioni è diventata una 
preoccupazione primaria. La crittografia, l'arte di proteggere le informazioni attraverso 
la loro codifica, gioca un ruolo fondamentale in questo contesto. Tra le varie branche 
della crittografia moderna, la crittografia basata sui reticoli emerge come un campo di 
studio particolarmente promettente e innovativo.
I reticoli, strutture matematiche composte da punti regolarmente distribuiti nello spazio 
n-dimensionale, offrono una base solida per la costruzione di schemi crittografici. La 
loro importanza risiede in alcuni problemi matematici intrinsechi, che risultano 
essere difficili da risolvere anche per computer quantistici. Questi ultimi infatti 
rappresentano una minaccia crescente per molti sistemi crittografici tradizionali 
che, essendo stati pensati per la forza computazionale odierna, non sono in grado di 
resistere ai ben più potenti calcolatori quantistici già in sviluppo attualmente. 
In questo contesto, il sistema crittografico GGH (Goldreich-Goldwasser-Halevi), 
proposto nel 1997, rappresenta una pietra miliare. GGH sfrutta la complessità 
computazionale di uno specifico problema sui reticoli, il CVP (Closest Vector Problem) 
per garantire la sicurezza delle informazioni. Tuttavia, come molti sistemi 
pionieristici, GGH ha mostrato alcune 
vulnerabilità nel corso degli anni, stimolando la ricerca di ottimizzazioni e 
miglioramenti. 
Una di queste ottimizzazioni è rappresentata da GGH-HNF (GGH - Hermite Normal Form), 
una variante che mira a rafforzare la sicurezza del sistema originale e cercare di risolverne 
le principali vulnerabilità. GGH-HNF introduce 
modifiche significative alla struttura della chiave pubblica, sfruttando la forma normale 
di Hermite per rendere il sistema più resistente a determinati tipi di attacchi.
Nonostante GGH e GGH-HNF siano stati precedentemente dichiarati praticamente 
inutilizzabili a causa delle loro prestazioni considerate non impiegabili nella pratica, 
questo studio affronta come i recenti progressi in hardware e software possano potenzialmente 
migliorarne le performance, rivalutandone potenzialmente la rilevanza nel contesto 
tecnologico attuale. Questa tesi si propone quindi di rivisitare entrambi i crittosistemi con 
strumenti moderni, offrendo un'analisi approfondita che comprende la spiegazione 
delle loro proprietà matematiche, l'esposizione degli algoritmi necessari al loro 
funzionamento e una crittoanalisi dettagliata. 
Quest'analisi teorica viene dunque configurata in un pacchetto Python 
strutturato in tre moduli: due dedicati specificamente ai crittosistemi esaminati, e un 
terzo che raccoglie funzioni e metodi comuni, oltre a strumenti utili nell'ambito della 
crittografia basata sui reticoli. Viene inoltre presentata, oltre alle implementazioni originale, 
anche una proposta di miglioramento attraverso una versione ibrida, la quale
mira a superare alcune delle limitazioni identificate nei sistemi originali. 
Successivamente ad un'analisi approfondita e la discussione dei risultati sperimentali ottenuiti,
questo studio si pone l'obiettivo di rispondere alla seguente domanda cruciale: GGH e le sue varianti 
potrebbero essere usate in ambito crittografico pratico? La valutazione finale mirerà quindi a 
fornire una prospettiva concreta sull'applicabilità di questi sistemi crittografici 
nel contesto tecnologico attuale, considerando sia i loro punti di forza che le loro 
potenziali limitazioni.