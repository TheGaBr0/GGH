\chapter{Introduzione}
\label{cap1}

Nell'era digitale odierna, la sicurezza delle informazioni è diventata una priorità 
cruciale. La crittografia, in particolare quella basata sui reticoli, svolge un ruolo 
fondamentale in questo contesto. I reticoli offrono una solida resistenza matematica 
anche ai computer quantistici, una minaccia emergente per molti sistemi crittografici 
tradizionali.
Il crittosistema a chiave pubblica GGH, proposto nel 1997, rappresenta una pietra miliare 
in questo tipo di 
crittografia, fondandosi sulla complessità posta da uno in particolare dei problemi più famosi
sui reticoli, ovvero il Problema del Vettore più corto (CVP).
Tuttavia, lo schema originale ha mostrato vulnerabilità, stimolando la ricerca di miglioramenti 
come la variante ottimizzata GGH-HNF.
Questa tesi si propone dunque di rivisitare approfonditamente GGH e GGH-HNF, valutandone la 
potenziale applicabilità pratica nel contesto tecnologico attuale. L'obiettivo di questa tesi è dunque 
quello di rivedere i crittosistemi menzionati in precedenza utilizzando strumenti moderni, mediante 
un'implementazione dettagliata supportata da: un'attenta analisi delle loro proprietà 
matematiche, la descrizione degli algoritmi necessari al loro funzionamento e una crittoanalisi 
approfondita.\\
Il Capitolo 2 si concentra sulle proprietà e i problemi relativi ai reticoli. 
Viene innanzitutto presentata una panoramica dettagliata sui reticoli, tra cui la 
definizione formale, le nozioni base come la proprietà dei coefficienti integrali, e il 
concetto del dominio fondamentale. Vengono inoltre introdotti i principali problemi 
reticolari di interesse crittografico, ovvero il Problema del Vettore più Corto (SVP) e 
il CVP. Il capitolo prosegue illustrando algoritmi 
chiave per la riduzione di basi reticolari. Viene prima presentato l'algoritmo d'ortogonalizzazione
Gram-Schmidt, un passo preliminare essenziale per il successivo processo di riduzione. In seguito, 
viene introdotto l'algoritmo di Lenstra-Lenstra-Lovász (LLL) e alcune sue versioni perfezionate. 
Questi algoritmi permettono di trasformare 
basi "cattive" in basi "buone", migliorando il rapporto di Hadamard che misura la qualità 
della base.
Infine, il capitolo si concentra su algoritmi per la risoluzione approssimata del CVP, 
come gli algoritmi di Babai e la tecnica di incorporamento, che sfruttano basi reticolari 
ridotte per ottenere soluzioni efficaci. \\
Il Capitolo 3 descrive e crittoanalizza il crittosistema a chiave pubblica Goldreich 
Goldwasser Halevi (GGH), oggetto di questa tesi. Viene spiegato in dettaglio il funzionamento 
di GGH, con particolare focus sulla generazione delle chiavi pubblica e privata. 
Il capitolo analizza poi le principali vulnerabilità del sistema GGH originale, 
classificandoli per efficacia e proponendone eventuali
contromisure. Il capitolo inoltre contiene al suo interno esempi pratici e dettagliati del funzionamento 
di GGH e di alcuni dei suoi attacchi principali. \\
Il Capitolo 4 affronta la variante migliorata dello schema originale, nota come GGH-HNF 
Questa versione mira a risolvere le 
vulnerabilità della versione originale di GGH aumentandone sia le performance che la 
sicurezza. Il capitolo descrive in dettaglio la struttura e il funzionamento di GGH-HNF, 
spiegando come la scelta d'utilizzo della forma normale di Hermite (HNF) possa aver apportato 
migliorie significative. 
Il capitolo procede poi riportando i limiti pratici di GGH-HNF e discutendone la sua sicurezza, 
presentando inoltre un esempio pratico del funzionamento dello schema. \\
Il Capitolo 5 descrive come l'implementazione, configurata come pacchetto Python, sia stata pensata e 
costruita. Vengono innanzitutto argomentate le ragioni che hanno guidato specifiche scelte progettuali, 
accompagnate da una descrizione dettagliata delle soluzioni implementate e delle ottimizzazioni apportate 
per migliorare le prestazioni del sistema. Il capitolo procede poi con la descrizione della
struttura del pacchetto, basata su tre moduli: i primi due dedicati ai crittosistemi singoli e il 
terzo contenente metodi in comune e funzioni utili nell'ambito della crittografia basata sui reticoli. 
Ogni modulo viene affrontato singolarmente nel dettaglio, spiegando: parametri in input, funzioni e 
decisioni implementative. \\
Il Capitolo 6 offre una panoramica completa dei risultati sperimentali ottenuti attraverso una serie 
di test mirati sull'intero progetto. L'analisi si concentra su tre aspetti chiave: i tempi di esecuzione,
l'occupazione di memoria per chiavi e testi cifrati, e il livello di sicurezza raggiunto. Inoltre, 
il capitolo introduce una variante ibrida che combina elementi dei due crittosistemi originali, 
mirando a ottenere prestazioni potenzialmente superiori pur mantenendo le caratteristiche fondamentali 
di entrambi i sistemi di base. Il capitolo si conclude con un'analisi comparativa che mette a confronto 
i risultati 
ottenuti con due importanti riferimenti: gli studi precedenti condotti sugli stessi crittosistemi 
e le analisi di sistemi crittografici classici come RSA ed ElGamal. Questa comparazione fornisce un 
contesto più ampio per valutare l'efficacia e l'innovazione dei risultati presentati sia nel rispetto
di dati passati che nel rispetto di sistemi attualmente in uso e dalla dimostrata validità.\\ 
Il Capitolo 7, infine, discute le conclusioni ottenute dai risultati sperimentali, fornendo un punto 
di vista oggettivo per quanto riguarda l'usabilità di questi crittosistemi con le tecnologie attuali. 