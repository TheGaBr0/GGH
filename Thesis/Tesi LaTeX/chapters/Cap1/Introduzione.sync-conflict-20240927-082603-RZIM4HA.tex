\chapter{Introduzione}
\label{cap1}

Nell'era digitale odierna, la sicurezza delle informazioni è diventata una priorità 
cruciale. La crittografia, in particolare quella basata sui reticoli, svolge un ruolo 
fondamentale in questo contesto. I reticoli offrono una solidità matematica e resistenza 
anche ai computer quantistici, una minaccia emergente per molti sistemi crittografici 
tradizionali.
Il sistema GGH, proposto nel 1997, rappresenta una pietra miliare nella crittografia 
basata sui reticoli, sfruttando la complessità del problema dei vettori più vicini sui reticoli. Tuttavia, il sistema originale GGH ha mostrato vulnerabilità, stimolando la ricerca di miglioramenti come la variante GGH-HNF.
Questa tesi si propone di rivisitare approfonditamente GGH e GGH-HNF, valutandone la 
potenziale applicabilità pratica nel contesto tecnologico attuale. \\
Il Capitolo 2 si concentra sulle proprietà e i problemi relativi ai reticoli. 
Viene innanzitutto presentata una panoramica dettagliata sui reticoli, tra cui la 
definizione formale, le nozioni base come la proprietà dei coefficienti integrali, e il 
concetto del dominio fondamentale. Vengono inoltre introdotti i principali problemi 
reticolari di interesse crittografico, ovvero il Problema del Vettore più Corto (SVP) e 
il Problema del Vettore più Vicino (CVP). Il capitolo prosegue illustrando algoritmi 
chiave per la riduzione di basi reticolari, introducendo prima l'ortogonalizzazione
Gram-Schmidt necessaria per la riduzione e successivamente l'algoritmo di Lenstra-Lenstra-Lovász 
(LLL) e sue varianti migliorate. Questi algoritmi permettono di trasformare 
basi "cattive" in basi "buone", migliorando il rapporto di Hadamard che misura la qualità 
della base.
Infine, il capitolo si concentra su algoritmi per la risoluzione approssimata del CVP, 
come gli algoritmi di Babai e la tecnica di incorporamento, che sfruttano basi reticolari 
ridotte per ottenere soluzioni efficaci. \\
Il Capitolo 3 descrive e crittoanalizza il crittosistema a chiave pubblica Goldreich 
Goldwasser Halevi (GGH), oggetto di questa tesi. Viene spiegato in dettaglio il funzionamento 
di GGH, con particolare focus sulla generazione delle chiavi pubblica e privata. 
Il capitolo analizza poi le principali vulnerabilità del sistema GGH originale, come 
gli attacchi di calcolo della chiave privata, risoluzione diretta del CVP, l'attacco di 
Nguyen che sfrutta la struttura del vettore di errore, e gli attacchi basati su 
informazioni parziali del messaggio. Per ciascun attacco sono forniti esempi pratici e 
discusse le possibili contromisure proposte. \\
Il Capitolo 4 presenta la variante migliorata del sistema GGH, nota come GGH-HNF, 
proposta da Daniele Micciancio nel 2001. Questa versione mira a risolvere le 
vulnerabilità della versione originale di GGH aumentandone sia le performance che la 
sicurezza. Il capitolo descrive in dettaglio la struttura e il funzionamento di GGH-HNF, 
spiegando come Micciancio abbia scelto di utilizzare la forma normale di Hermite (HNF) 
per generare la chiave pubblica invece della costruzione casuale originale. 
In questo capitolo vengono anche riportati anche i dati riguardanti i limiti pratici di 
GGH-HNF, rilevati da un rapporto tecnico di Christoph Ludwig nel 2004. \\
Il Capitolo 5 esplora l'implementazione dei crittosistemi GGH e GGH-HNF, strutturata 
come un pacchetto Python. Questa sezione offre una dettagliata giustificazione delle 
scelte progettuali e presenta un'analisi approfondita dei tre moduli principali che 
compongono il pacchetto: GGH e GGH-HNF, che implementano rispettivamente i due sistemi 
in esame, e Utils, un modulo di supporto che fornisce funzioni e metodi di 
utilità generale utilizzati da entrambi i crittosistemi. \\
Il Capitolo 6 espone i risultati sperimentali dei test condotti sui sistemi crittografici 
precedentemente discussi. Presenta dati dettagliati sui tempi di esecuzione per ciascuna 
fase operativa e sulle dimensioni delle chiavi e dei testi cifrati. L'analisi si conclude 
con un confronto tra questi risultati e quelli ottenuti da studi precedenti sugli stessi 
crittosistemi, nonché con schemi crittografici tradizionali come RSA ed ElGamal. \\
Il Capitolo 7, infine, discute i risultati ottenuti e risponde alle domande poste dalla 
tesi, con particolare attenzione ai vantaggi e, soprattutto, gli svantaggi rilevati. 