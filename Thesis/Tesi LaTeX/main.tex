\DocumentMetadata{pdfversion=2.0, pdfstandard=A-4}
\documentclass[a4paper,12pt]{report}
\usepackage{graphicx}
\graphicspath{{images/}}

\usepackage{blindtext}
\usepackage{subfiles}
\usepackage[a4paper]{geometry}
\usepackage{amssymb,amsmath,amsthm,amsfonts}
\usepackage{url}
\usepackage{hyperref}
\usepackage{epsfig}
\usepackage[italian]{babel}
\usepackage{setspace}
\usepackage{tesi}
\usepackage{mathtools}
\usepackage{caption}
\usepackage{subcaption}
\usepackage{tikz}
\usetikzlibrary{backgrounds}
\usepackage{float}
\usepackage{algorithm2e}
\RestyleAlgo{ruled}
\usepackage{amsthm}
\theoremstyle{definition}
\newtheorem{exmp}{Esempio}[section]


% per le accentate
\usepackage[utf8]{inputenc}
\setlength\parindent{0pt}
%per gli algoritmi in italiano
\renewcommand{\algorithmcfname}{Algoritmo}%

%hyperref figure%
\newcommand*{\figref}[2][]{%
  \hyperref[{fig:#2}]{%
    Figure~\ref*{fig:#2}%
    \ifx\\#1\\%
    \else
      \,#1%
    \fi
  }%
}


%
%
%			TITOLO
%
\begin{document}
\title{Crittografia post-quantistica basata sui reticoli: Implementazione e crittoanalisi di GGH}
\author{Gabriele BOTTANI}
\dept{Corso di Laurea in Sicurezza dei sistemi e delle reti informatiche} 
\anno{2023-2024}
\matricola{01701A}
\relatore{Prof.\ Stelvio CIMATO}

% 
%			DEDICA
%
\beforepreface
\afterpreface
% 
% 
%			CAPITOLO 1: Introduzione
\subfile{chapters/Cap1/Introduzione.tex}
%
%			CAPITOLO 1: Teoria
\subfile{chapters/Cap2/Teoria.tex}
%
%		    CAPITOLO 2: GGH
\subfile{chapters/Cap3/GGH.tex}
%
%		    CAPITOLO 3: GGH-HNF
\subfile{chapters/Cap4/GGH-HNF.tex}
%
%			BIBLIOGRAFIA
%
\begin{thebibliography}{}
    \bibitem{GGHMayNot15}
    de Barros Charles Figueredo e Menasché Schechter Luis, “GGH May Not Be Dead after All,” 
    Proceeding Series of the Brazilian Society of Computational and Applied Mathematics, 
    21941-590 Rio de Janeiro RJ, 2015
    \bibitem{Galbraith18}
    Galbraith Steven, “Mathematics of Public Key Cryptography”, seconda edizione, Ottobre 2018
    \bibitem{Nguyen99}
    Nguyen Phong, “Cryptanalysis of the Goldreich-Goldwasser-Halevi Cryptosystem from 
    Crypto ’97,” Advances in Cryptology — CRYPTO’ 99, 45 rue d’Ulm, 75230 Paris Cedex 05, 
    France, pagine 288–304, 1999
    \bibitem{Babai86}
    Babai László, “On Lovász’ Lattice Reduction e the Nearest Lattice Point Problem,” 
    Combinatorica, vol. 6, no. 1, pagine 1–13, 1986
    \bibitem{GGH97}
    Goldreich Oded, Goldwasser Shafi e Halevi Shai,  “Public-key Cryptosystems from 
    Lattice Reduction Problems,” Advances in Cryptology — CRYPTO ’97, pagine 112–131, 1997
    \bibitem{HNF01}
    Micciancio Daniele,  “Improving Lattice Based Cryptosystems Using the Hermite 
    Normal Form,” Lecture Notes in Computer Science, 9500 Gilman Drive, La Jolla, 
    CA 92093 USA, pagine 126–145, 2001
    \bibitem{PQC09}
    Micciancio Daniele e Regev Oded, “Lattice-based Cryptography,” Post-Quantum 
    Cryptography, pagine 147–191, 2009
    \bibitem{CVP-NP09}
    Aharonov Dorit e Regev Oded, “Lattice Problems in NP $\cap$ coNP“, 
    CiteSeer X (The Pennsylvania State University), 2009
    \bibitem{SVP-NP02}
    Micciancio Daniele e Goldwasser Shafi, 
    “Complexity of Lattice Problems: a cryptographic perspective“,
    The Kluwer International Series in Engineering and Computer Science, Boston, Massachusetts,
    Kluwer Academic Publishers, volume 671, 2002
    \bibitem{HDMRD08}
    Silverman Joseph H., Pipher Jill e Hoffstein Jeffrey, “An introduction to mathematical cryptography“,
    seconda edizione, Springer, Undergraduate texts in mathematics, 2008
    \bibitem{LLL82}
    Lenstra Arjen Klaas, Lenstra Hendrik Willem e László Lovász, 
    “Factoring polynomials with rational coefficients“, Mathematlsche Annalen, Springer, 
    volume 261, pagine 515-534, 1982
    \bibitem{FPLLL05}
    Nguyen Phong e Damien Stehlé, “Floating-point LLL revisited“, LNCS, Springer, 
    volume 3494, pagine 215-233, 2005
    \bibitem{DEEPLLL94}
    Schnorr Claus Peter e M. Euchner, 
    “Lattice basis reduction: Improved practical algorithms and solving subset sum problems“,
    Mathematical Programming, volume 66, pagine 181-199, 1994
    \bibitem{BKZPRUNED}
    Schnorr Claus Peter e H. H. Hörner, 
    “Attacking the Chor-Rivest cryptosystem by improved lattice reduction“,
    Proc. of Eurocrypt'95, Springer-Verlag, volume 921, pagine 1-12, 1995
    \bibitem{Nguyen400}
    Moon Sung Lee e Sang Geun Hahn, “Cryptanalysis of the GGH Cryptosystem“,
    Mathematics in Computer Science, volume 3, pagine 201-208, 2010

   
\end{thebibliography}
% 
\end{document}