Desidero esprimere la mia gratitudine a diverse persone che mi hanno assistito, in maniera
diretta o indiretta, nel processo di stesura di questo elaborato.  
In primo luogo, rivolgo un sincero ringraziamento al mio relatore, il professor 
Stelvio Cimato, per avermi offerto l'opportunità di intraprendere questa tesi e per 
il suo prezioso supporto durante tutto il suo sviluppo. Un secondo ringraziamento va ai 
miei amici, in particolare quelli che ho conosciuto durante i miei tre anni di università
e quelli che invece conosco da una vita. La vostra presenza è stata importante perchè è 
grazie a voi se vivo sempre con il sorriso e con la tranquillità di avere sempre qualcuno 
su cui contare e con cui distrarmi nei momenti in cui ne ho bisogno. 
Non posso tralasciare, sebbene possa sembrare scontato, di esprimere la mia più profonda 
gratitudine anche ai miei genitori, Serena e Antonio. Il loro costante supporto e incoraggiamento sono stati 
pilastri fondamentali nel mio percorso. Senza di loro, non solo non sarei qui a 
scrivere queste parole, ma non avrei nemmeno avuto la possibilità di conseguire 
questo traguardo accademico. Il loro ruolo è stato determinante non solo nel rendermi 
la persona che sono oggi, ma anche nel permettermi di raggiungere questo importante 
obiettivo formativo.
Un doveroso ringraziamento va anche ai revisori, il cui contributo è stato fondamentale
per il perfezionamento di questo lavoro. La loro attenzione ai dettagli e
i loro suggerimenti hanno notevolmente arricchito la qualità
dell'elaborato, permettendomi di affinare le mie argomentazioni e di presentare un
lavoro più completo e rigoroso. Filippo, conosciuto relativamente da poco, ma che ha 
fornito un contributo fondamentale nella precisazione e correzione a livello matematico; 
mio zio Vittorino, per il suo aiuto nella riformulazione di frasi e nella
correzione della struttura logico-tecnica della tesi.
Un ringraziamento speciale va a quest'ultimo non solo per avermi messo a 
disposizione la macchina su cui sono stati eseguiti tutti gli esperimenti presentati, 
senza la quale non avrei potuto fornire risultati così precisi e validi.
Devo inoltre riconoscere che è grazie a lui se sono arrivato a questo punto del mio
percorso: la sua passione per l'informatica non solo mi ha dato l'ispirazione necessaria
per intraprendere questa strada, ma mi ha anche fornito un modello da seguire.
Un ultimo ringraziamento, ma indubbiamente non per importanza, va alla mia ragazza, la 
mia Chia. Lei è stata la fonte costante di supporto e incoraggiamento durante tutto il 
mio percorso da quasi tre anni a questa parte, la persona con cui ho potuto sfogarmi e allo stesso tempo 
festeggiare. La sua pazienza, comprensione e il suo amore incondizionato sono stati 
fondamentali per superare questi anni e lo saranno anche per i prossimi avvenire.  
Un'ultima volta grazie a tutti voi che avete contribuito in modi diversi ma ugualmente 
preziosi al mio cammino. Questo traguardo non è solo mio, ma il risultato del sostegno, 
dell'amicizia e della compagnia che ognuno di voi mi ha fornito. Grazie di cuore.